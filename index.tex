% Options for packages loaded elsewhere
\PassOptionsToPackage{unicode}{hyperref}
\PassOptionsToPackage{hyphens}{url}
\PassOptionsToPackage{dvipsnames,svgnames,x11names}{xcolor}
%
\documentclass[
  letterpaper,
  DIV=11,
  numbers=noendperiod]{scrreprt}

\usepackage{amsmath,amssymb}
\usepackage{iftex}
\ifPDFTeX
  \usepackage[T1]{fontenc}
  \usepackage[utf8]{inputenc}
  \usepackage{textcomp} % provide euro and other symbols
\else % if luatex or xetex
  \usepackage{unicode-math}
  \defaultfontfeatures{Scale=MatchLowercase}
  \defaultfontfeatures[\rmfamily]{Ligatures=TeX,Scale=1}
\fi
\usepackage{lmodern}
\ifPDFTeX\else  
    % xetex/luatex font selection
\fi
% Use upquote if available, for straight quotes in verbatim environments
\IfFileExists{upquote.sty}{\usepackage{upquote}}{}
\IfFileExists{microtype.sty}{% use microtype if available
  \usepackage[]{microtype}
  \UseMicrotypeSet[protrusion]{basicmath} % disable protrusion for tt fonts
}{}
\makeatletter
\@ifundefined{KOMAClassName}{% if non-KOMA class
  \IfFileExists{parskip.sty}{%
    \usepackage{parskip}
  }{% else
    \setlength{\parindent}{0pt}
    \setlength{\parskip}{6pt plus 2pt minus 1pt}}
}{% if KOMA class
  \KOMAoptions{parskip=half}}
\makeatother
\usepackage{xcolor}
\setlength{\emergencystretch}{3em} % prevent overfull lines
\setcounter{secnumdepth}{5}
% Make \paragraph and \subparagraph free-standing
\ifx\paragraph\undefined\else
  \let\oldparagraph\paragraph
  \renewcommand{\paragraph}[1]{\oldparagraph{#1}\mbox{}}
\fi
\ifx\subparagraph\undefined\else
  \let\oldsubparagraph\subparagraph
  \renewcommand{\subparagraph}[1]{\oldsubparagraph{#1}\mbox{}}
\fi


\providecommand{\tightlist}{%
  \setlength{\itemsep}{0pt}\setlength{\parskip}{0pt}}\usepackage{longtable,booktabs,array}
\usepackage{calc} % for calculating minipage widths
% Correct order of tables after \paragraph or \subparagraph
\usepackage{etoolbox}
\makeatletter
\patchcmd\longtable{\par}{\if@noskipsec\mbox{}\fi\par}{}{}
\makeatother
% Allow footnotes in longtable head/foot
\IfFileExists{footnotehyper.sty}{\usepackage{footnotehyper}}{\usepackage{footnote}}
\makesavenoteenv{longtable}
\usepackage{graphicx}
\makeatletter
\def\maxwidth{\ifdim\Gin@nat@width>\linewidth\linewidth\else\Gin@nat@width\fi}
\def\maxheight{\ifdim\Gin@nat@height>\textheight\textheight\else\Gin@nat@height\fi}
\makeatother
% Scale images if necessary, so that they will not overflow the page
% margins by default, and it is still possible to overwrite the defaults
% using explicit options in \includegraphics[width, height, ...]{}
\setkeys{Gin}{width=\maxwidth,height=\maxheight,keepaspectratio}
% Set default figure placement to htbp
\makeatletter
\def\fps@figure{htbp}
\makeatother
% definitions for citeproc citations
\NewDocumentCommand\citeproctext{}{}
\NewDocumentCommand\citeproc{mm}{%
  \begingroup\def\citeproctext{#2}\cite{#1}\endgroup}
\makeatletter
 % allow citations to break across lines
 \let\@cite@ofmt\@firstofone
 % avoid brackets around text for \cite:
 \def\@biblabel#1{}
 \def\@cite#1#2{{#1\if@tempswa , #2\fi}}
\makeatother
\newlength{\cslhangindent}
\setlength{\cslhangindent}{1.5em}
\newlength{\csllabelwidth}
\setlength{\csllabelwidth}{3em}
\newenvironment{CSLReferences}[2] % #1 hanging-indent, #2 entry-spacing
 {\begin{list}{}{%
  \setlength{\itemindent}{0pt}
  \setlength{\leftmargin}{0pt}
  \setlength{\parsep}{0pt}
  % turn on hanging indent if param 1 is 1
  \ifodd #1
   \setlength{\leftmargin}{\cslhangindent}
   \setlength{\itemindent}{-1\cslhangindent}
  \fi
  % set entry spacing
  \setlength{\itemsep}{#2\baselineskip}}}
 {\end{list}}
\usepackage{calc}
\newcommand{\CSLBlock}[1]{\hfill\break\parbox[t]{\linewidth}{\strut\ignorespaces#1\strut}}
\newcommand{\CSLLeftMargin}[1]{\parbox[t]{\csllabelwidth}{\strut#1\strut}}
\newcommand{\CSLRightInline}[1]{\parbox[t]{\linewidth - \csllabelwidth}{\strut#1\strut}}
\newcommand{\CSLIndent}[1]{\hspace{\cslhangindent}#1}

\KOMAoption{captions}{tableheading}
\makeatletter
\@ifpackageloaded{bookmark}{}{\usepackage{bookmark}}
\makeatother
\makeatletter
\@ifpackageloaded{caption}{}{\usepackage{caption}}
\AtBeginDocument{%
\ifdefined\contentsname
  \renewcommand*\contentsname{Table of contents}
\else
  \newcommand\contentsname{Table of contents}
\fi
\ifdefined\listfigurename
  \renewcommand*\listfigurename{List of Figures}
\else
  \newcommand\listfigurename{List of Figures}
\fi
\ifdefined\listtablename
  \renewcommand*\listtablename{List of Tables}
\else
  \newcommand\listtablename{List of Tables}
\fi
\ifdefined\figurename
  \renewcommand*\figurename{Figure}
\else
  \newcommand\figurename{Figure}
\fi
\ifdefined\tablename
  \renewcommand*\tablename{Table}
\else
  \newcommand\tablename{Table}
\fi
}
\@ifpackageloaded{float}{}{\usepackage{float}}
\floatstyle{ruled}
\@ifundefined{c@chapter}{\newfloat{codelisting}{h}{lop}}{\newfloat{codelisting}{h}{lop}[chapter]}
\floatname{codelisting}{Listing}
\newcommand*\listoflistings{\listof{codelisting}{List of Listings}}
\makeatother
\makeatletter
\makeatother
\makeatletter
\@ifpackageloaded{caption}{}{\usepackage{caption}}
\@ifpackageloaded{subcaption}{}{\usepackage{subcaption}}
\makeatother
\ifLuaTeX
  \usepackage{selnolig}  % disable illegal ligatures
\fi
\usepackage{bookmark}

\IfFileExists{xurl.sty}{\usepackage{xurl}}{} % add URL line breaks if available
\urlstyle{same} % disable monospaced font for URLs
\hypersetup{
  pdftitle={Fullstack-ai},
  pdfauthor={Norah Jones},
  colorlinks=true,
  linkcolor={blue},
  filecolor={Maroon},
  citecolor={Blue},
  urlcolor={Blue},
  pdfcreator={LaTeX via pandoc}}

\title{Fullstack-ai}
\author{Norah Jones}
\date{2024-02-20}

\begin{document}
\maketitle

\renewcommand*\contentsname{Table of contents}
{
\hypersetup{linkcolor=}
\setcounter{tocdepth}{2}
\tableofcontents
}
\bookmarksetup{startatroot}

\chapter*{Preface}\label{preface}
\addcontentsline{toc}{chapter}{Preface}

\markboth{Preface}{Preface}

This is a Quarto book.

To learn more about Quarto books visit
\url{https://quarto.org/docs/books}.

\bookmarksetup{startatroot}

\chapter{Introduction}\label{introduction}

This book is a collection of notes, tutorials, and examples employed for
the Fullstack AI course at the Autonoma de Occidente University in Cali,
Colombia. The course has been designed to provide a comprehensive
introduction to the development process of AI systems, focusing on the
practical aspects. The course was created to be accessible to students
with a wide range of backgrounds, including students without prior AI
experience.

The course was created and developed by Dr.~Henry Ruiz, who currently
works as a research scientist at Texas A\&M AgriLife Research. The
course is based on the author's experience in developing AI systems for
a wide range of applications, including agriculture, healthcare, and
finance.

\bookmarksetup{startatroot}

\chapter{Curriculum}\label{curriculum}

\begin{enumerate}
\def\labelenumi{\arabic{enumi}.}
\item
  \textbf{Setting up end-to-end AI projects}: Just like any software
  solution, ML systems require a well-structured methodology to ensure
  high success rates. The challenge often lies not in the ML algorithms
  themselves, but in integrating these algorithms with the rest of the
  software and hardware components of the system to solve real-world
  problems. It's noted that 60 out of 96 failures are due to non-ML
  components, and 60\% of models never reach production. This section
  discuss some of the challenges encountered in AI system development
  and provides a guide for building AI systems. This guide includes
  defining the problem, gathering data, evaluating different ML
  methodologies, and integrating the model into an AI system.
\item
  \textbf{Defining the stack technology}: The number of available tools
  to work with ML seems endless, and selecting the appropiate tools
  depends on the kind of problem, type of solution, deployment model,
  capacity building, team experience, cost, hardware and software
  infrastructure, etc. This section will discuss different tools used in
  production to develop and implement ML systems and how they can be
  integrated.
\item
  \textbf{Data Management}: Data is the most important asset in AI
  systems. The quality of the data will determine the quality of the
  model. This section will discuss the best practices for managing data
  in AI systems, including data collection, data cleaning, data storage,
  and data processing. It will also discuss how to select the appropiate
  data management tools for data storage, data processing, and data
  visualization.
\item
  \textbf{Machine Learning Teams}: Machine Learning talents are
  expensive and scarce, and machine Learning teams have diverse roles.
  Managing and leading ML and Data Science teams require unique skills.
  Here we will learn more about these roles' importance and impact
  within the organization.
\item
  \textbf{Training debuging and desin patterns for ML solutions}: This
  section will discuss the best practices for training ML models,
  including data preprocessing, model selection, model training, and
  model evaluation. It will also discuss how to debug ML models and how
  to use design patterns to build scalable and maintainable ML
  systems.In engineering disciplines, design patterns capture best
  practices and solutions to commonly occurring problems. They codify
  the knowledge and experience of experts into advice that all
  practitioners can follow.
\item
  \textbf{Testing and Deployment}: A machine learning model can only
  begin to add value to an organization when that model's insights
  routinely become available to the users for which it was built. The
  process of taking a trained ML model and making its predictions
  available to users or other systems is known as deployment. Let's
  learn together about troubleshooting and deploying ML models in
  production and how we can ship ML models to production using different
  deployments strategies and scenarios.
\item
  \textbf{ML operation: DevOps -\textgreater{} MLOps -\textgreater{}
  LLMOps -\textgreater{} FMOps:} The development of ML systems is a
  complex process that requires the collaboration of different teams,
  including data scientists, software engineers, and operations teams.
  This section will discuss the challenges of integrating ML models into
  production systems and the best practices for managing ML models in
  production.MLOps is a methodology for ML engineering that unifies ML
  system development (the ML element) with ML system operations (the Ops
  element). It advocates formalizing and (when beneficial) automating
  critical steps of ML system construction. This course will discuss how
  MLOps maximize the capacities and resources of ML teams by providing a
  set of standardized processes and technology capabilities for
  building, deploying, and operationalizing ML systems rapidly and
  reliably
\end{enumerate}

\bookmarksetup{startatroot}

\chapter{Summary}\label{summary}

In summary, this book has no content whatsoever.

\bookmarksetup{startatroot}

\chapter*{References}\label{references}
\addcontentsline{toc}{chapter}{References}

\markboth{References}{References}

\phantomsection\label{refs}
\begin{CSLReferences}{0}{1}
\end{CSLReferences}



\end{document}
